\section*{Project Abstract}

Fixed-pitch quadcopters has become increasingly popular over the past years and have become the platform of choice for researchers and hobbyists. The platform is preferred due to its mechanical simplicity and robustness compared to other flying platforms. 
\\\\
Lately, there has been a significant reduction in cost related to quadcopters. The hardware has become cheaper, smaller and carries more functionality and processing power then ever before. This has made the technology more widely available and has contributed to a leap in advances and new applications. 
\\\\
Beside all the research, advances in technology and the inherent simplicity, the fixed-pitch quadrotor has its limitations. A fixed-pitch quadrotor achieves stability and flight control by changing the speed of the individual motors by a increase or decrease in voltage over the motors. Changing the rotational speed of the motor is not an instantaneous process and requires time to accelerate. The motors has to overcome the inertia of the rotating parts to get the propellers spinning faster, or it has to dissipate the kinetic energy stored in the rotation to go slower. Flight control based solely on changing motor speeds works well in many applications, but has shown to be inadequate in turbulent and difficult flying conditions, especially when landing. 
\\\\
The question, is it possible to improve the capabilities of quadcopters by implementation of variable pitch. With variable pitch actuation, thrust can be changed almost instantaneously only limited by the speed of the servo actuation. With variable pitch, the simple quadcopter may overcome its limitations of instability in challenging flying conditions and turbulent landings. 


\begin{comment}
Quadcopters are currently a hot topic in today’s society and in technological development. They have a wide range of applications and can perform a variety of tasks, such as aerial photography, acrobatic flying, transportation of goods and many more. Quadcopters are currently used by military, law enforcement, hobbyists and for commercial applications. 
\\\\
Over the last few years we have seen a significant reduction in cost related to quadcopters. This has made the technology more widely available and has contributed to a leap in 
advances and new applications. But beside all the research done, there is marginal documentation on the benefits of quadcopters using variable pitch systems.
\\\\
From helicopters we know that variable pitch is essential for maneuverability. The purpose of the variable pitch system is to shift the thrust produced by the propeller blades quickly, generating the required lift. 
\\\\
In comparison, most multicopters have fixed pitch propellers. This means that their movement is solely controlled by accelerating or decelerating individual propellers to change position. Changing position in this way has in some cases shown to be slow and stability in disturbing and turbulent conditions is a problem. 
\\\\
But what if a multicopter had the ability to control both pitch and RPM? Would this yield superior agility and maneuverability, and what effects will this have on efficiency? 
\\\\
On this basis Norwegian Research Establishment(FFI) has requested that we investigate the advantages and disadvantages of variable pitch quadcopters. 
\end{comment}