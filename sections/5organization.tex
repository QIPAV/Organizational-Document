\section{Organization and Tools}

This project follows the Scrum framework. See section about Scrum to learn more about our project model. Meetings are performed in conformance with the Scrum ceremonies. Management software is used to facilitate Scrum and to track and plan progress. JIRA is the software of choice, containing the backlog, sprint backlog and burndown charts, as well as resources. Additionally, a project plan is generated, represented by a Gantt-chart to give an overview and ensure that the project deadline is met. Project cycles are fixed-length iterations, referred to as Sprints. The sprints have planned endings and starts every other Tuesday, starting February $17^{th}$ $2017$. Human resources are allocated in JIRA as each sprint is defined in the sprint planning meeting.
\\\\
ShareLaTex is also used. LaTeX is a high-quality typesetting system; it includes features designed for the production of technical and scientific documentation. LaTeX is the "de facto" standard for the communication and publication of scientific documents \cite{latex}. ShareLaTex is the version of the software that allows for editing documents online with multiple participants simultaneously.  
\\\\
In order to have all documents saved at all times, the documents are pushed to GitHub. GitHub is a development platform inspired by the way you work. Host code, manage projects, and build software alongside millions of other developers. GitHub brings teams together to work through problems, move ideas forward, and learn from each other along the way \cite{github}. This guarantees that all documents are saved, in the case of any unforeseen situations that can cause loss of documents. It also ensures traceability of all document history. 

\subsection{Work Hours}

The group has agreed that working hours are between 8:15 to 16:15, Tuesday to Friday. This gives a total of 32 planned work hours per person in one week. After Easter, the team will have one additional day, since exams are over. One day a week is a nonworking day, preferably Saturday. Sundays will be utilized if additional work days are required. 

\subsection{Meetings}
In the second week of a sprint, Tuesday from 10:00 to 11:00 the sprint retrospective meeting is held. The team discuss and evaluate the last sprint, and learn from mistakes and successes. The wisdom gained in the retrospective meeting is noted and considered in the Sprint Planning meeting.
\\\\
After the retrospective meeting, Sprint Planning meeting is held from 11:00 to 14:00. In this meeting, user-stories are given a relative time estimate called  story-points. Stories are divided into  tasks and chores that must be completed in order to achieve the sprint goal. The outcome is a product backlog with elaborated and estimated tasks. The team can take on any task or chore they see fit to work on. 
\\\\
A Spring Log is also made after each Sprint. The Sprint Log describes what have been done during this Sprint, what have not been done, why is wasn't done and what could have been done differently. This allows the team to learn from each Sprint and to make improvements for the upcoming and following Sprints. 
\\\\
Scrum reviews are performed after a sprint is ended. The planned days for these meetings are Fridays. The attendees are the team, product owner, scrum master, internal and/or external supervisors. Product increments produced in the last sprint is presented and discussed, any necessary changes and re-prioritization are made.
\\\\
Once a day, the group performs a daily stand-up meeting of no more than 15 minutes. This meeting is referred to as "Daily Scrum" and starts 8:45 AM. 









