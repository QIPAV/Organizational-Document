\section{The Project}

Norwegian Defence Research Establishment(FFI), has the need to examine some properties regarding quadcopters with variable pitch. \\
\\
A quadcopter with traditional fixed-pitch propellers changes position by accelerating or decelerating individual propellers. Thus, generating more or less thrust as required.
But in some cases this method has shown to be slow, and the quadcopter can't be held stable in disturbing and turbulent conditions.\\
\\
For this project the Norwegian Defence Research Establishment(FFI) want the students to:
\begin{itemize}
    \item Build a small ($<2,5kg$) quadcopter with variable pitch propellers.
    \item Measure the response-time of the variable pitch propeller quadcopter and compare it to traditional fixed pitch propeller quadcopter. 
    \item Investigate if there is a possibility of more stable flight and landing. 
\end{itemize}

\subsection{Project Outline} 
In order to answer the questions posed by FFI, a quadcopter must be designed, built and then tested. Building a quadcopter is an interdisciplinary effort, containing elements from software, hardware and electrical engineering. \\
\\
The team has to acquire a rigorous understanding of quadcopters from theory to practice. Two quadcopters must be built, a controller system must be developed, and the control system has to be translated into logic software.\\
\\
Testing and flights shall be performed in Kongsberg Innovation Center(KIC) by the use of a tracking system called Qualisys. Most computations will be done on a external computer based on the outputs from the tracking system. On the quadcopter, there will be a device that computes the inner loop that is responsible for the stabilisation. It also receives data from the external computer and makes new calculations based on these data.\\
\\
When a working prototype is produced, there will be performed a series of tests to acquire the necessary data to answer the questions posed by FFI.

